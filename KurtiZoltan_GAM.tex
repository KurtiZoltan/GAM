\documentclass[pdftex,12pt,a4paper]{article}
\pdfpagewidth 8.5in
\pdfpageheight 11.6in
\linespread{1.3}
\usepackage{anysize}
\marginsize{2.5cm}{2.5cm}{2.5cm}{2.5cm}

\usepackage[utf8]{inputenc}
\usepackage[T1]{fontenc}
\usepackage[english]{babel}
\usepackage{indentfirst}
\usepackage{amsmath}
\usepackage{float}
\usepackage{graphicx}
\usepackage{braket}
\usepackage[unicode,pdftex]{hyperref}
%\usepackage{hyperref}
\usepackage{breqn}

\usepackage{listings}
\usepackage{xcolor}

\definecolor{codegreen}{rgb}{0,0.6,0}
\definecolor{codegray}{rgb}{0.3,0.3,0.3}
\definecolor{codepurple}{rgb}{0.58,0,0.82}
\definecolor{backcolour}{rgb}{0.90,0.90,0.87}

\lstdefinestyle{mystyle}{
    backgroundcolor=\color{backcolour},   
    commentstyle=\color{codegreen},
    keywordstyle=\color{magenta},
    numberstyle=\small\color{codegray},
    stringstyle=\color{codepurple},
    basicstyle=\ttfamily\small,
    breakatwhitespace=false,         
    breaklines=true,                 
    captionpos=b,                    
    keepspaces=true,                 
    numbers=left,                    
    numbersep=5pt,                  
    showspaces=false,                
    showstringspaces=false,
    showtabs=false,                  
    tabsize=2
}
\lstset{style=mystyle}

\DeclareMathOperator{\Ai}{Ai}
\DeclareMathOperator{\Bi}{Bi}
\DeclareMathOperator{\Aip}{Ai^\prime}
\DeclareMathOperator{\Bip}{Bi^\prime}
\DeclareMathOperator{\Ti}{Ti}
\DeclareMathOperator{\ctg}{ctg}
\DeclareMathOperator{\sgn}{sgn}
%\DeclareMathOperator{\max}{max}
\let\Im\relax
\DeclareMathOperator{\Im}{Im}
\DeclareMathOperator{\Tr}{Tr}
\newcommand{\op}[1]{\hat{#1}}
\newcommand{\norm}[1]{\left\lVert #1 \right\rVert}
\newcommand*\Laplace{\mathop{}\!\mathbin\bigtriangleup}

\newcommand{\aeqref}[1]{\az{\eqref{#1}}}
\newcommand{\Aeqref}[1]{\Az{\eqref{#1}}}

\hypersetup{
    colorlinks,
    citecolor=black,
    filecolor=black,
    linkcolor=black,
    urlcolor=black
}
\hypersetup{	
	pdftitle={Gamma spectroscopy},
	pdfauthor={Kürti Zoltán}}

\frenchspacing
\begin{document}

	\centerline{\bf\LARGE Gamma spectroscopy}

	\vskip0.4truein\centerline{\Large\sc Kürti Zoltán}\vskip0.10truein
	%\centerline{\includegraphics[scale=0.5]{./elte_cimer_color.pdf}}
	\vskip0.4truein
	\centerline{\Large Group B}\vskip0.2truein
	\centerline{\Large{Measurement date: 2021.11.04.}}\vskip0.2truein
	\centerline{\Large{Hand in date: 2021.11.15.}}\vskip0.2truein
	\thispagestyle{empty}
	\newpage
	\tableofcontents
	\newpage
	\section{Introduction}
	\subsection{}
	
	\section{Granite sample}
	
	\section{Soil sample}
	
	
	
	
		%\begin{figure}[H]
			%\centering
			%\includegraphics[scale=1]{./figs/1dpsd.pdf}
			%\caption{\Aref{fftlines}. ábrán látható piros és kék irányokhoz tartozó spektrum abszolútérték négyzete, logaritmikus skálán, a narancssárga függőleges vonalak a periódushoz tartozó hullámszám egész számú többszörösei.}
		%	\label{1dpsd}
		%\end{figure}
		%Ahogy azt \aref{1dpsd}. ábra mutatja, a két irányban vett power spektrum konzisztens a négyzetráccsal, azaz csúcsaik pozíciói megegyeznek. A leolvasott $K$ érték $K=49.3\frac{1}{mm}$. Az ehhez tartozó rácsállandó $L=127\mu m$.
	\section{Konklúzió}
	\bibliographystyle{abeld}
    \bibliography{ref}
\end{document}